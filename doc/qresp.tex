\documentclass[reprint,prd, superscriptaddress, tightenlines, longbibliography, nofootinbib, eqsecnum, amsfonts, amsmath, floatfix, notitlepage, twocolumn]{revtex4-1}
%\documentclass{article}
\usepackage{pgfplots}
\usepackage{hyperref}

\usepackage[a4paper, total={7.1in, 9in}]{geometry}
\usepackage{amsmath} 
\usepackage{amssymb}
\usepackage[utf8]{inputenc}
\usepackage{color}
\newcommand{\T}[0]{{\mathcal T}}
\newcommand{\Xd}[0]{{X^{\rm dat}}}
\newcommand{\Xwf}[0]{{X^{\rm WF}}}
\newcommand{\Cov}[0]{{\textrm{Cov}}}

\newcommand{\si}[0]{{s_{\rm i}}}
\newcommand{\ti}[0]{{t_{\rm i}}}
\newcommand{\ui}[0]{{u_{\rm i}}}
\newcommand{\vi}[0]{{v_{\rm i}}}
\newcommand{\so}[0]{{s_{\rm o}}}
\renewcommand{\to}[0]{{t_{\rm o}}}
\newcommand{\uo}[0]{{u_{\rm o}}}
\newcommand{\vo}[0]{{v_{\rm o}}}

\newcommand{\sr}[0]{{s_{\rm r}}}
\newcommand{\tr}[0]{{t_{\rm r}}}
\newcommand{\ur}[0]{{u_{\rm r}}}
\renewcommand{\vr}[0]{{v_{\rm r}}}

\newcommand{\Ylm}[1]{\:_{#1}Y_{\ell m}}
\newcommand{\Ylms}[1]{\:_{#1}Y^*_{\ell m}}
\newcommand{\YLM}[1]{\:_{#1}Y_{L M}}
\newcommand{\YLMs}[1]{\:_{#1}Y^*_{L M}}
\newcommand{\av}[1]{\left\langle #1 \right\rangle}
\newcommand{\resp}{ {\mathcal R} }

\newcommand{\red}[1]{\color{red}#1 \color{black} }
\newcommand{\JC}[1]{\color{red}JC: #1\color{black}}
\newcommand{\hn}[0]{\hat n}
\begin{document}
\title{Notes on curved-sky quadratic estimation}
\begin{abstract}This document supplements the release of the Planck 2018 CMB lensing \cite{Aghanim:2018oex} pipeline. It collects the formulae relevant to curved-sky quadratic estimators in the spin-weight, position-space formalism, including in particular estimator cross-responses and Gaussian noise biases between arbitrary pairs of quadratic estimators. \JC{Document to be included with the pipeline release after submission of the revised L08. in progress}

\end{abstract}	
\author{Julien Carron}

\maketitle
\tableofcontents

\vspace{1cm}

%\section{Notes on curved-sky quadratic estimation}
The purpose of this document is to collect the formulae relevant to CMB lensing and other quadratic estimators in the spin-weight, position-space formalism. 

The gradient (g) and curl (c) modes of definite parity of a complex spin-$r$ field $_{r}\alpha(\hn)$ (defined by the condition that it transforms under a rotation of angle $\psi$ of the local axes through $_{r}\alpha(\hn) \rightarrow e^{i r \psi}\:_{r}\alpha(\hn)$ ) are defined through
\begin{eqnarray}
		g^{r}_{LM} &= -\frac 12\left(\:_{|r|} \alpha_{LM} + (-1)^r \:_{-|r|} \alpha_{LM}\right)\\
		c^{r}_{LM} &=-\frac 1{2i} \left( \:_{|r|} \alpha_{LM} - (-1)^r \:_{-|r|} \alpha_{LM} \right)
\end{eqnarray}
where  $_{\pm r} \alpha_{LM} \equiv \int d^2n \:_{\pm r}\alpha(\hn) \:_{\pm r}Y^*_{LM}(\hn)$ (we adopt the convention, standard in CMB lensing, to write quadratic estimator multipoles with $L, M$ and use $\ell, m$ for the CMB fields from which they are built).
The inverse relation is
\begin{equation}
	_{\pm |r|} \alpha_{LM} =- (\pm)^r\left( g^r_{LM} \pm i c^r_{LM} \right).
\end{equation}
Prior to projection onto gradient and curl modes, and prior to proper normalization, quadratic estimators can be written in the form \begin{equation}\label{QE}
\begin{split}	
 _{r}\hat \alpha(\hn) \equiv &\left(\sum_{\ell m}\: w^{\so\si}_\ell \:_{\si} \bar X_{\ell m} \:_\so Y_{\ell m}(\hn)\right)\\\cdot &\left(\sum_{\ell m}\:w^{\to\ti}_\ell  \:_{\ti} \bar X_{\ell m} \:_\to Y_{\ell m}(\hn)\right)
 \end{split} 
\end{equation}
where $\si, \ti$ are input spins, $\so, \to$ outputs spins, and $w_\ell^{\so\si}, w_\ell^{\ti\to}$ associated weights. Obviously, $s_o + t_o = r$, and by consistency with $_{-r} \alpha(\hn) = \:_r\alpha^*(\hn)$ the weights have symmetry $w_\ell^{-\so-\si} = (-1)^{\so + \si}w_\ell^{*\so \si}$.
\newline
\newline
The maps $_s \bar X_{lm}$ are the inverse variance filtered CMB maps; the filtered scalar temperature
\begin{equation}
	_0 \bar X_{\ell m} = \bar T_{\ell m}
\end{equation}	
and filtered spin $\pm 2$ Stokes polarization $_{\pm 2}P = \bar Q \pm i\bar U$,
\begin{equation}
\quad _{\pm 2} \bar X_{\ell m} = _{\pm 2}\bar P_{\ell m}= -\left(\bar E_{\ell m} \pm i\bar B_{\ell m} \right),
\end{equation}
(for the purposes of the analytical calculations in this document) are isotropically related to the (beam-deconvolved) data maps $_sX^{\rm dat}$ through a matrix $F$,
\begin{equation}\label{eq:filter}
	_{s}\bar X_{\ell m} \equiv \sum_{s_2 = 0,2,-2}F_\ell^{s s_2} \:_{s_2}X_{\ell m}
\end{equation} 
(isotropic approximation of $\bar X = \mathcal B^\dagger \Cov^{-1} X ^\textrm{\rm dat}$ in the notation of Ref.~\cite{Aghanim:2018oex}
\begin{equation}\label{eq:cf}
\begin{split}
	\xi^{st}_{+}(\beta) &\equiv \av{e^{-is \alpha}\:_{s}X(\hn_1)\:\left(_{t}X(\hn_2)e^{-i t\gamma}\right)^*}\\
		\xi^{st}_{-}(\beta)& \equiv \av{ \left(e^{-is \alpha}\:_{s}X(\hn_1)\right)^*\:\left(_{t}X(\hn_2)e^{-i t\gamma}\right)^*}
\end{split}
\end{equation}
$\gamma$ is the angle at $\hn_1$ that aligns the local $x$-axis to the geodesic connecting $\hn_1$ and $\hn_2$ (with the axis pointing towards $\hn_2$), $\beta$ the angle between $\hn_1$ and $\hn_2$, and $\alpha$ is defined just as $\gamma$ but at $\hn_2$.\cite{Chon:2003gx, Challinor:2005jy}
\JC{Mixups with n1 and n2 defs...fix this!}
\begin{widetext}
\begin{equation}
\begin{split}
	\xi_{+}^{st}(\beta) =\left(+1\right)^s &\sum_{L} \left(\frac{2L + 1}{4\pi}\right) \left[C_L^{g^sg^{t}} + C_L^{c^sc^{t} }-i\left(C_L^{g^sc^{t}} + C_L^{c^sg^{t}}\right)\right]d^L_{s t}(\beta) \\
	\xi_{-}^{st}(\beta) = \left(-1\right)^s &\sum_{L} \left(\frac{2L + 1}{4\pi}\right) \left[C_L^{g^sg^{t}} - C_L^{c^sc^{t} }-i\left(C_L^{g^sc^{t}} - C_L^{c^sg^{t}}\right)\right]d^L_{-s t}(\beta) 
\end{split}
\end{equation}
\end{widetext}


%from http://background.uchicago.edu/~whu/tamm/webversion/node5.html
%For independently filtered temperature and polarization such as the Planck 2018 baseline analysis, the filtered $\bar T, \bar E, \bar B$ are directly proportional to $T, E$ and $B$ respectively, with spin-space matrix $F$ in Eq.~\eqref{eq:filter}
%\begin{equation}
%	F = \begin{pmatrix}
%		F^T_\ell & 0 & 0 \\ 0 & \frac 12 \left( F^{E}_\ell  + F^{B}_\ell\right) & \frac 12 \left( F^{E}_\ell  - F^{B}_\ell\right) \\ 0& \frac 12 \left( F^{E}_\ell  - F^{B}_\ell\right) & \frac 12 \left( F^{E}_\ell  + F^{B}_\ell\right)
%	\end{pmatrix}
%\end{equation}
%where
%\begin{equation}
%F_\ell^{X} = \frac{1}{C_\ell^{XX,\rm fid} + N_\ell^{\rm X} /b_\ell^2}, \quad X = T,E,B.
%\end{equation}
%In Ref~\cite{Aghanim:2018oex}, $F_\ell^{X}$ is set to zero outside $100 \le \ell \le 2048$, $N^T_\ell$ is $35 \mu$K-amin, $N^P_\ell$ is $55 \mu$K-amin, $b_\ell$ is Gaussian beam of FWHM $5$-amin, and $F_\ell^X$ contains further an additional small rescaling. 
%For joint temperature and polarization filtering, the $F$ matrix becomes:
%\begin{equation}
%	F = \begin{pmatrix}
%		F^{TT}_\ell &  -\frac 12 F^{TE} &   -\frac12 F^{TE} \\  -F^{TE} & \frac 12 \left( F^{EE}_\ell  + F^{B}_\ell\right) & \frac 12 \left( F^{EE}_\ell  - F^{B}_\ell\right) \\ -F^{TE}& \frac 12 \left( F^{EE}_\ell  - F^{B}_\ell\right) & \frac 12 \left( F^{EE}_\ell  + F^{B}_\ell\right)
%	\end{pmatrix}
%\end{equation}
%where the entries $F^{T, E, B}$ are the elements of
%\begin{equation}
% \begin{pmatrix}  C_\ell^{TT} + N_\ell^T &C^{TE}_\ell &  0 \\ C^{TE}_\ell & C_\ell^{EE} + N_\ell^E  & 0\\  0 & 0 &  C_\ell^{BB} + N_\ell^B
% 	
% \end{pmatrix}^{-1}
%\end{equation}
%\newline
%\newline
The formulae exposed in this document can be derived through simple application of this formal relation,
\begin{widetext}
\begin{equation}
\begin{split}
&\sum_{m_1,m_2}\int d^2n_1\:_{s_1} Y_{\ell_1 m_1}(\hn_1)\:_{s_2} Y_{\ell_2 m_2}(\hn_1)\:_{r_1} Y_{L M}(\hn_1)\int d^2n_2\:_{t_1} Y_{\ell_1 m_1}(\hn_2)\:_{t_2} Y_{\ell_2 m_2}(\hn_2)\:_{r_2} Y_{L' M'}(\hn_2)  \\&= \delta_{LL'}\delta_{MM'}\frac{2\ell_1 + 1}{4\pi}\frac{2\ell_2 + 1} {4\pi} 2\pi \int_{-1}^{1} d\beta \: d^{\ell_1}_{s_1,t_1}(\beta)d^{\ell_2}_{s_2 t_2}(\beta)d^{L}_{r_1 r_2}(\beta) \quad (\textrm{whenever } s_1 + s_2 + r_1  = 0 = t_1 + t_2 + r_2).
\end{split}
\end{equation}
\end{widetext}
where $d^\ell_{mm'}$ are Wigner small d-matrices.
\begin{figure}[h]
	\includegraphics[width=0.5\textwidth]{HuHarmonics3.png}
	\caption{\label{fig:geometry}The geometry and angles in Eq.~\eqref{eq:cf}, with the local axes in green. It holds $\alpha(\hn_2, \hn_1) = \pi - \gamma(\hn_1, \hn_2)$ and $ \gamma(\hn_2, \hn_1) = \pi - \alpha(\hn_1, \hn_2)$. Figure originally from Wayne Hu tutorials, \url{http://background.uchicago.edu/~whu/tamm/webversion/node5.html}.}
\end{figure}
\subsection{Gaussian covariance calculations}
Q.E. noise covariance can be evaluated with a series of one-dimensional integrals as was first demonstrated by Ref.~\cite{}. For two generic estimators as defined in Eq.~\eqref{QE}, we now obtain their gradient (g) and curl (c) covariances with four integrals as follows.

For an isotropy estimator $_{r}\hat \alpha$ let $s = (\si, \so, w^{\si\so})$ collectively describes the in and out spins and weight function of the left leg, and similarly with $t$ for the right leg (with $\so + \to = r$). In the same way, let $u$ and $v$ describes another estimator $_{r'}\hat \alpha$ (with $\uo + \vo = r'$). Then, their Gaussian correlation functions are
\begin{equation}\boxed{
	\xi^{rr'}_{\pm}(\beta) = \xi^{\pm s, u}(\beta) \xi^{\pm t, v}(\beta) +  \xi^{\pm s, v}(\beta) \xi^{\pm t, u}(\beta)},
\end{equation}

 %may be written $ \left.\av{\:_{r}\hat \alpha_{LM}\: _{r'}\hat \alpha^*_{L'M'}} \right|_{\rm Gauss} \equiv \delta_{LL'}\delta_{MM'}n_L^{stuv}$ with

%\begin{equation}
%\boxed{
%\begin{split} 
%n_L^{stuv} & = (-1)^{r + r'}2\pi  \int_{-1}^1 d \mu\:  d^L_{-r -r'}(\mu)% \left[\xi^{su}(\mu)\:\xi^{tv}(\mu)  + \xi^{sv}(\mu)\:\xi^{tu}(\mu) \right]
%\end{split}}
%\end{equation}

where $\xi^{s,t}$ is
\begin{equation}\boxed{
\xi^{s,t}(\mu) \equiv  \sum_\ell \left(\frac{2\ell + 1}{4\pi}\right)w^{\so\si}_\ell w^{*\to\ti}_\ell \bar C_\ell^{\si \ti} d^\ell_{\so\to}(\mu)}
\end{equation}
and $\bar C_\ell^{\si \ti} \equiv \av{ _{\si}\bar X_{\ell m}\: _{\ti} \bar X^*_{\ell m} }$.
Projecting onto gradient and curl modes results in
%\begin{equation} \boxed{
%\begin{split}
%\left.\av{\hat g^{r}_{LM} \hat g^{*, r'}_{L' M'} }\right|_{\rm Gauss.} &=\delta_{LL'}\delta_{MM'} \frac 12 \Re\left[n_L^{stuv} +  (-1)^{r} n^{-s-tuv}_L\right] \\
%		\left.\av{\hat c^{r}_{LM} \hat c^{*, r'}_{L' M'} }\right|_{\rm Gauss.} &= \delta_{LL'}\delta_{MM'}\frac 12 \Re\left[n_L^{stuv} -  (-1)^{r} n^{-s-tuv}_L\right]\\
%	\left.\av{\hat g^{r}_{LM} \hat c^{*, r'}_{L' M'} }\right|_{\rm Gauss.} &= \delta_{LL'}\delta_{MM'}\frac 12 \Im\left[-n_L^{stuv} -  (-1)^{r} n^{-s-tuv}_L\right] \\ \left.\av{\hat c^{r}_{LM} \hat g^{*, r'}_{L' M'} }\right|_{\rm Gauss.} &= \delta_{LL'}\delta_{MM'}\frac 12 \Im\left[n_L^{stuv} -  (-1)^{r} n^{-s-tuv}_L\right]
%\end{split}}
%\end{equation}
\begin{equation}
\begin{split}
\left.\av{\hat g^{r}_{LM} \hat g^{*, r'}_{L' M'} }\right|_{\rm G.} &=\delta_{LL'}\delta_{MM'} \frac 12 \Re\left[C_L^{rr'} +  (-1)^{r} C_L^{-rr'}\right] \\
		\left.\av{\hat c^{r}_{LM} \hat c^{*, r'}_{L' M'} }\right|_{\rm G.} &= \delta_{LL'}\delta_{MM'}\frac 12 \Re\left[C_L^{rr'} -  (-1)^{r} C_L^{-rr'}\right]\\
	\left.\av{\hat g^{r}_{LM} \hat c^{*, r'}_{L' M'} }\right|_{\rm G.} &= \delta_{LL'}\delta_{MM'}\frac 12 \Im\left[-C_L^{rr'} -  (-1)^{r} C_L^{-rr'}\right] \\ \left.\av{\hat c^{r}_{LM} \hat g^{*, r'}_{L' M'} }\right|_{\rm G.} &= \delta_{LL'}\delta_{MM'}\frac 12 \Im\left[C_L^{rr'} -  (-1)^{r} C_L^{-rr'}\right]
\end{split}
\end{equation}
where \begin{equation}
C_L^{\pm rr'}  \equiv 2\pi  \int_{-1}^1 d \mu\:  d^L_{\pm rr'}(\mu) \xi^{rr'}_{\pm}(\beta)
\end{equation}
($\Re$ and $\Im$ stands for real and imaginary parts respectively). 
Ref.~\cite{Aghanim:2018oex} calculates the covariance matrix based on these equations using the empirical, realisation dependent power spectra $\bar C_\ell^{s_i,t_i}$. A gradient-curl mode cross-covariance  may be sourced by gradient-curl couplings in the inverse-variance filtered CMB fields (i.e., non-zero $C_\ell^{\bar T \bar B}$ or $C_\ell^{\bar E \bar B}$). In most relevant situations there is no such couplings and the gradient to curl and curl to gradient covariance vanish.


\subsection{Response and cross-responses calculations}
We now turn to the calculation of the response of the estimator to a source of anisotropy. Anisotropy can sometimes be parametrized at the level of the CMB maps, (for example for lensing), with
\begin{equation}\label{eq:mapresp}
	_{s}\delta X(\hn) = \sum_{a = \pm r}\:_{a}\alpha(\hn) \left( \sum_{\ell m}\: R_\ell^{a, s} \:_sX_{\ell m} \Ylm {s- a}(\hn)\right)
\end{equation}
for response kernel functions $R^{r,s}_\ell$. More generally, let the covariance of the CMB data respond as follows to a spin-$r$ anisotropy source $\alpha$:
\begin{widetext}
\begin{equation}\label{eq:covresp}
	\delta  \av{_sX(\hn_1) \:_tX^*(\hn_2)} =   \sum_{\ell m, a = \pm r}\:_{a}\alpha(\hn_1) W_\ell^{a, st} \:_{s - a}Y_{\ell m}(\hn_1)  \:_{t}Y^*_{\ell m}(\hn_2)  +   W_\ell^{* a, ts} \:_{s}Y_{\ell m}(\hn_1)  \:_{t-a}Y^*_{\ell m}(\hn_2)\:_{a}\alpha^*(\hn_2)
\end{equation}
\end{widetext}
for some weights functions $W_\ell^{a, st}$. For map-level descriptions in Eq.~\eqref{eq:mapresp} then holds
\begin{equation}
	W_\ell^{a, st} = R^{a, s} C_\ell^{st}.
\end{equation}
However, Eq.~\eqref{eq:covresp} is more general.
Examples include:
\begin{enumerate}
	\item Lensing\cite{Okamoto:2003zw}: The source of anisotropy is the spin-1 field $_1\alpha(\hn)$, with linear response (see Ref.~\cite{Challinor:2002cd})
	$\delta _sX(\hn) =  -\frac 12 \alpha_1(\hn) \bar \eth _{s}X(\hn) - \frac 12 \alpha_{-1}(\hn) \eth \:_sX(\hn) $
	where $\eth$ and $\bar \eth$ are the spin raising and spin lowering operator respectively. Hence
	\begin{equation}
	\begin{split}	
		R_\ell^{-1, s} &=- \frac 12\sqrt{ (l - s) (l + s + 1) } \\
		R_\ell^{1, s} &= +\frac12\sqrt{ (l + s) (l - s + 1) }
	\end{split}
	\end{equation}
	\item CMB modulation: The source is a scalar, with response $\delta _sX(\hn) = \:_0\alpha(\hn) _{s}X(\hn)	$, hence
	\begin{equation} 
	R_\ell^{0,s} = 1
	\end{equation}
	\item Point sources in temperature ($S^2$, see Ref.~\cite{Osborne:2013nna}): here anisotropy is sought of the form
	$\delta  \av{T(\hn) \:T(\hn')} = \delta_{\hat n\hat n'}S^2(\hn)$. Hence,
	\begin{equation}
	W^{r, st}_\ell = \frac 14\delta_{r0}\delta_{s0}\delta_{t0} 
	\end{equation}
	\item Polarization rotation (for example from polarization angle miscalibration). There the observed polarization is rotated according to $_{\pm 2} X$ is $e^{\mp 2i \:_{0}\alpha} \: _{\pm 2}X$. Hence,
	\begin{equation}
		 R_\ell^{0, \pm 2} = \mp 2i
	\end{equation}
	\item Noise variance map anisotropies (basically the same as point sources but acting on beam-convolved maps)	\begin{equation}
	W^{r, st}_\ell = \frac 14\delta_{r0}\delta_{s0}\delta_{t0}  \frac{1}{b_\ell^2}
	\end{equation}
\end{enumerate}


Let as before $s, t$ denote collectively the QE spins and weight functions for an estimator $_r\hat \alpha(\hn)$ of spin $r = s_o + t_o$, and let $r'$ be the spin of anisotropy source $_{r'}\beta(\hn)$ with covariance response kernel $W^{r'}$ as above. Let $\mathcal R_L^{g_r g_{r'}} \delta_{LL'}\delta_{MM'}$ be defined as the response of the gradient mode of $\alpha_{LM}$ to the gradient mode of $\beta_{L'M'}$, and similarly for the curl. It holds: \begin{equation}%\boxed{
	\begin{split}
		\resp^{g_rg_{r'}}_L &= \Re\left[R_L^{st, r'} + (-1)^{r'} R_L^{st, -r'}\right]\\
		\resp^{c_rc_{r'}}_L &= \Re\left[R_L^{st, r'} - (-1)^{r'} R_L^{st, -r'}\right] \\
		\resp^{g_rc_{r'}}_L &= \Im\left[-R_L^{st, r'} + (-1)^{r'} R_L^{st, -r'} \right]  \\
		\resp^{c_rg_{r'}}_L &= \Im\left[R_L^{st, r'} + (-1)^{r'} R_L^{st, -r'} \right] \\
	\end{split}
	%}
\end{equation}
where
\begin{widetext}
\begin{equation}
%\boxed{
\begin{split}
R_L^{st, r'} &= 2\pi  \int_{-1}^1 d \mu\: d^L_{rr'}(\mu)\sum_{\tilde s_i,\tilde t_i = 0,2,-2}  \left[\xi^{\so \si \tilde s_i} (\mu)\psi^{\to \ti \tilde t_i \tilde s_i, r' }(\mu) +  \xi^{\to \ti \tilde t_i}(\mu) \psi^{\so \si \tilde s_i \tilde t_i, r' }(\mu) \right]
\end{split}
%}
\end{equation}
and
\begin{equation}
%\boxed{
\begin{split}
\xi^{\so\si \tilde s_i}(\mu) &\equiv  \sum_\ell \left(\frac{2\ell + 1}{4\pi}\right)w^{\so\si}_\ell F_\ell^{\si \tilde \si} d^\ell_{\so,\tilde \si}(\mu) 
\\
\psi^{\so\si \tilde \si \tilde \ti, r'}(\mu) &\equiv \sum_\ell \left(\frac{2\ell + 1}{4\pi}\right)w^{\so \si}F^{\si \tilde \si}_\ell W_\ell^{*-r', -\tilde \ti \tilde \si} d^\ell_{\so,-\tilde \ti + r'}(\mu) 
\end{split}
%}
\end{equation}
	
\end{widetext}
Again, in most relevant cases, the gradient to curl and curl to gradient responses do vanish. If there is a unique source of anisotropy, properly normalized gradient and curl estimators are then given by $\hat g^r_{LM} / \mathcal R_L^{g_rg_r}$ and $\hat c^r_{LM} / \mathcal R_L^{c_r c_r}$.

\subsection{Derivation of optimal QE weights}Optimal (in the sense of minimal Gaussian variance) QE weights are easily gained from the representation in Eq.~\ref{eq:covresp} of the anisotropy. Let the CMB likelihood gradients be
\begin{equation}
	_{\pm r}\hat \alpha(\hn) = \left.\frac{\delta }{\delta _{\mp r}\alpha (\hn)} -\frac 12\: _{s_1}X \Cov^{-1}_{s_1s_2} \:_{s_2}X \right|_{\alpha \equiv 0}
\end{equation}
where $\Cov_{s_1 s_2}(\hn, \hn') \equiv \av{_{s_1}X^{}(\hn) \:_{s_2}X^{}(\hn') }$, and where $_r\alpha(\hn)$ and $_{-r}\alpha(\hn)$ are treated as independent variables. 
Using Eq.~\eqref{eq:covresp} and comparing to Eq.~\eqref{QE}, we find
\begin{equation}\boxed{
	w_\ell^{st} = \delta_{st} \textrm{   (1st leg)  } \quad 	w_\ell^{-s + r, t} = 2W^{-r, -st}_\ell \textrm{   (2nd leg)  }}
\end{equation} \JC{why 2 again? }
\JC{FIXME: The right expression is 
\begin{equation}
	_{r}\hat g(\hn) = \sum_{s}\: _{-s}\bar X(\hn)\cdot \left( 2W_{\ell}^{-r, st} \:_{t}\bar X_{\ell m}\: _{s + r}Y_{\ell m}(\hn)\right)
\end{equation}
where $\bar X$ has the $(0, 2, -2)$ elements (note the additional factor of 2! in pol w.r.t. to naive spin defs.)
\begin{equation}
\begin{pmatrix}
	\bar T \\ -\frac 12 \left( \bar E + i \bar B \right) \\-\frac 12 \left( \bar E - i \bar B \right)
\end{pmatrix}	
\end{equation}
Factor of 2 in front of W comes from 2 $ \delta/  \delta\: _{-r}\alpha(\hn)$ to get $d/dre + d/dim$ (?).}
\begin{figure}
	\includegraphics[width=0.5\textwidth]{../figs/planck_lensing_no}
	\caption{\label{fig:plancklensing_no}Lensing gradient and curl reconstruction noise levels for a \textit{planck}-like experiment.}
\end{figure}

\bibliography{lensing}
\end{document}